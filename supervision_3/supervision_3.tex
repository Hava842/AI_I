\documentclass{exam}
\usepackage[utf8]{inputenc}
\usepackage{amsmath}
\usepackage{hyperref}
 
\begin{document}
\noindent
\large\textbf{Artificial Intelligence 3} \hfill Supervisior: Marton Havasi \\
\normalsize Lectures 9-12ish \hfill 20/05/2018

\paragraph{Core Questions}
\begin{questions}

\question Partial Order Planning: 2016 paper 4 question 1 \href{http://www.cl.cam.ac.uk/teaching/exams/pastpapers/y2016p4q1.pdf}{Link}

\question Problem sheet 8.1 

\question Learning: 2013 paper 4 question 2 \href{http://www.cl.cam.ac.uk/teaching/exams/pastpapers/y2013p4q2.pdf}{Link} 

\question The MNIST database of handwritten digits has a training set of 60,000 examples, and a test set of 10,000 examples. \href{http://yann.lecun.com/exdb/mnist/}{http://yann.lecun.com/exdb/mnist/}

Your task is to implement and train a neural network to classify the written digits.

Your neural network will have two fully-connected hidden layers each with 100 nodes. The nodes will have a sigmoid activation function and your loss function should be the cross-entropy loss. Do not forget to include bias terms in each layer.

A comprehensive python tutorial can be found here \href{https://www.tensorflow.org/versions/r1.0/get_started/mnist/beginners}{Link}, but you may use any programming language of your choice. This tutorial does not include hidden layers so you will have to add those yourself.

Report the training and test accuracies.
\end{questions}

\paragraph{Tryhard questions (for exam preparation)}
\begin{questions}
\question Planning + CSP: 2017 paper 4 question 1 \href{http://www.cl.cam.ac.uk/teaching/exams/pastpapers/y2017p4q1.pdf}{Link} 

\end{questions}


\end{document}